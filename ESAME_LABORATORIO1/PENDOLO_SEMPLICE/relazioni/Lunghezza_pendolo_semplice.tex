\documentclass{article}

\usepackage[utf8]{inputenc}
\usepackage[italian]{babel}
\usepackage{subfigure}
\usepackage{float}
\usepackage{amssymb}
\usepackage{geometry}
\geometry{a4paper, top=1cm, bottom=2cm, left=1.5cm,right=1.5cm, marginparwidth=1.75cm}
\usepackage{amsmath}
\usepackage{graphicx}
\usepackage[colorlinks=true, allcolors=blue]{hyperref}
\usepackage{textcomp}
\usepackage{multicol}
\usepackage{booktabs}
\usepackage{siunitx}

\begin{document}
	\author{Nicolò Bottiglioni}
	\title{Periodo pendolo matematico}
	\date{3 Luglio 2023}
	\maketitle
	
	\section{Obiettivi}
	L'obiettivo dell'esperienza è verificare che il periodo di oscillazione di un pendolo semplice, fissato un angolo $\theta_{0}$, dipende dalla lunghezza l del pendolo stesso secondo la legge 
	\begin{equation}
		T=2\pi\sqrt{\frac{l}{g}}(1 + \frac{1}{16}\theta_{0}^{2} + \frac{11}{3072}\theta_{0}^{4} + ....) \quad [s]
	\end{equation}
	dove g è costante di gravità che, per l'esperienza, si assume pari a 9.81 $[\frac{m}{s^{2}}]$.
	
	\section{Apparato sperimentale}
	Erano a disposizione i seguenti strumenti
	\begin{itemize}
		\item cronometro di risoluzione 0.01 s;
		\item metro a nastro di risoluzione 1 mm;
		\item calibro ventesimale di risoluzione 0.05 mm;
		\item bilancia di precisione di risoluzione 0.001 g;
	\end{itemize}
	L'apparato sperimentale consiste in un filo, assunto come ideale quindi inestensibile e privo di massa, alla cui estremità era agganciata una massa. Il filo era arrotolato a due travi che permettevano di allungare oppure accorciare la lunghezza del pendolo stesso.
	
	\section{Misure effettuate}
	Per ogni diversa misura di lunghezza, il pendolo è stato fatto oscillare con la stessa ampiezza iniziale $\theta_{0}$. Quest'ultima è stata misurata 
\end{document}