\documentclass{article}

\usepackage[utf8]{inputenc}
\usepackage[italian]{babel}
\usepackage{subfigure}
\usepackage{float}
\usepackage{amssymb}
\usepackage{geometry}
\geometry{a4paper, top=1cm, bottom=2cm, left=1.5cm,right=1.5cm, marginparwidth=1.75cm}
\usepackage{amsmath}
\usepackage{graphicx}
\usepackage[colorlinks=true, allcolors=blue]{hyperref}
\usepackage{textcomp}
\usepackage{multicol}
\usepackage{booktabs}
\usepackage{siunitx}

\begin{document}
	\author{Nicolò Bottiglioni}
	\title{Indice di rifrazione del plexiglass}
	\date{3 Luglio 2023}
	\maketitle
	
	\section{Obiettivi}
	L'obiettivo dell'esperienza è quello di misurare l'indice di rifrazione del plexiglass a partire dalle misurazioni dell'angolo di incidenza e di rifrazione di un raggio luminoso incidente sulla superficie piana del plexiglass e sfruttando la legge di Snell-Cartesio:
	
	\begin{equation}
		 n_1sin\theta_i = n_2sin\theta_r \quad.
	\end{equation}
	 
	\section{Apparato sperimentale}
	L'apparato consiste in  un supporto per il semicilindro di plexiglass, una sorgente luminosa con un diaframma a fenditura che permette di creare un fascio di luce sottile, una lente convergente posizionata di fronte alla sorgente e un foglio circolare quadrettato sul quale sono evidenziati due diametri ortogonali della circonferenza.
	Il foglio è stato posizionato sul supporto e, grazie sia ai quadretti sia ai diametri ortogonali evidenziati, è stato possibile posizionare il solido di plexiglass in modo che il fascio di luce proveniente dalla sorgente incidesse nel centro della sua superficie piana. Prima di cominciare con le misurazioni, è stata regolata la distanza del supporto per il semicilindro dalla sorgente di luce in modo che il raggio incidente fosse il più sottile possibile.
	
	\section{Misure effettuate}
	Sono state misurate le distanze della normale al piano di incidenza rispettivamente del raggio incidente e del raggio rifratto. 
	Le coppie di misure effettuate sono state dunque, detto $R$ il raggio della circonferenza e $\theta_i$ e $\theta_r$ gli angoli di incidenza e rifrazioni, le seguenti.
	\vspace*{1em}
	\begin{center}
	\begin{tabular}{cc}
		\hline
	$Rsin\theta_r \pm \frac{1}{\sqrt{12}}  \quad [quadretti]$ & $Rsin\theta_i \pm \frac{1}{\sqrt{12}} \quad [quadretti]$ \\
		\hline
		&  \\
		&  \\
		&  \\
		&  \\
		&  \\
		&  \\
		&  \\
		&  \\
		&  \\
		&  \\
		\hline
	\end{tabular}
    \end{center}
    \vspace*{1em}
    
    Nelle misurazioni, la carta quadrettata ha svolto le veci di una sorta di metro a nastro di risoluzione pari ad un quadretto. Si è assunto quindi che le grandezze misurate fossero distribuite uniformemente in un intervallo di ampiezza pari ad un quadretto e, come incertezza di misura, è stata di conseguenza associata ad entrambe le misure $\frac{1}{\sqrt{12}} \quad[quadretti]$.
    
	\section{Analisi dati}
	A partire dalla $(1)$, si può esplicitare $sin\theta_i$ in funzione di $sin\theta_r$. Nel caso delle nostre misure:
	\vspace{1em}
	\begin{equation}
		n_1Rsin\theta_i=n_2Rsin\theta_r \quad\quad[quadretti]
	\end{equation}
	\begin{equation}
		sin\theta_i=\frac{n_2}{n_1}sin\theta_r \quad\quad[quadretti]
	\end{equation}
	\vspace{1em}
	Rispettivamente, $n_{1}$ e $n_{2}$ rapprsentano l'indice di rifrazione dell'aria e del plexiglass.
	E' stato eseguito un fit tramite l'algoritmo odr con un modello lineare, ottenendo il seguente risultato. 
	
	XXXXXX FIGURA XXXXXX
	
	
	Il fit tramite l'algoritmo odr è stato utilizzato perché l'incertezza sulla misura della distanza tra asse e raggio rifratto non è trascurabile.
	\\
	\\
	Inoltre, per quest'esperienza è stato assunto $n_1=1$, con $n_1$ indice di rifrazione dell'aria. L'effettivo valore dell'indice di rifrazione è, invece, circa $n_1=1,000294$. Con tale approssimazione si è introdotto un errore relativo pari a $ 3\cdot 10^{-4}$. Grazie a quest'approssimazione, il valore stimato dal fit per il coefficiente angolare coincide con il valore dell'indice di rifrazione del plexiglass.
	
	\section{Conclusioni}
	Il modello matematico utilizzato per il fit è quello di una retta, del tipo $y=mx+q$, come espresso dalla $(3)$. Stando a quest'ultima, i valori attesi per $n_2$, $q$ e $\chi^2$ dovrebbero essere i seguenti.
	
	\vspace{1em}
	
	\begin{center}
		\begin{tabular}{cc}
			\hline
			Grandezze & Valori attesi\\
			\hline
			$n_2$ & 1.48 \\
			$q$ &  0 \\
			$\chi^2$ &  8 $\pm$ 2.5 \\
			\hline
		\end{tabular}
	\end{center}
	
	\vspace{1em}
	
	Riportiamo di seguito i valori stimati tramite il fit.
	
	\vspace{1em}
	
	\begin{center}
		\begin{tabular}{cc}
			\hline
			Grandezze & Valori stimati \\
			\hline
			$n_2$ & $  \pm$  \\
			$q$ & $ \pm$  \\
			$\chi^2$ & $$ \\
			\hline
			
		\end{tabular}
	\end{center}
	
	\vspace{1em}
	
Notiamo che l'errore relativo sulla misura dell'indice di rifrazione del plexiglass è pari a $\frac{\sigma_{n_2}}{n_2} = XXXXX$. Essendo quest'ultima molto maggiore dell'errore relativo introdotto dall'approssimazione $n_{aria}=1$, l'approssimazione stessa è giustificata.

La misura ottenuta dell'intercetta risulta essere compatibile con lo zero, XXXXX.

Il valore del $\chi^{2}$ invece è/non è  XXXX compatiibile con il valore atteso. Va specificato che nel contesto di questo esperimento, il $\chi^{2}$ non è distribuito come un Chi quadro dal momento che esso è la somma di quadrati di variabili distribuite uniformemente e non di quadrati di variabili gaussiane. Pertanto, in questo caso il valore atteso del $\chi^{2}$ rimane comunque pari al numero di gradi di libertà $\nu = 10 - 2 = 8$, mentre la sua varianza è pari a $\frac{4\nu}{5}= \frac{32}{5}$, dove 4/5 è la varianza del quadrato di una variabile distribuita uniformemente, e di conseguenze la sua deviazione standard è $\sim 2,5$.
	

\end{document}